%%%%%%%%%%%%%%%%%%%%%%%%%%%%%%%%%%%%%%%%%
% Medium Length Professional CV
% LaTeX Template
% Version 2.0 (8/5/13)
%
% This template has been downloaded from:
% http://www.LaTeXTemplates.com
%
% Original author:
% Trey Hunner (http://www.treyhunner.com/)
%
% Important note:
% This template requires the resume.cls file to be in the same directory as the
% .tex file. The resume.cls file provides the resume style used for structuring the
% document.
%
%%%%%%%%%%%%%%%%%%%%%%%%%%%%%%%%%%%%%%%%%

%----------------------------------------------------------------------------------------
%	PACKAGES AND OTHER DOCUMENT CONFIGURATIONS
%----------------------------------------------------------------------------------------

\documentclass{resume} % Use the custom resume.cls style

\usepackage[left=0.75in,top=0.6in,right=0.75in,bottom=0.6in]{geometry} % Document margins
\usepackage{hyperref}

\name{Onur Şahin} % Your name
\address{Software Engineer} % Your address
\address{sahinonur2000@hotmail.com \\ https://github.com/onsah} % Your phone number and email

\begin{document}

%----------------------------------------------------------------------------------------
%	EDUCATION SECTION
%----------------------------------------------------------------------------------------

\begin{rSection}{Education}

{\bf Bilkent University, Turkey} \hfill {\em June 2021} \\ 
B.S. in Computer Science \\
Full Scholarship \\
Overall GPA: 3.49/4.00

{\bf Bilim ve Sanat Merkez, Turkey} \hfill {\em 2006-2016} \\ 
Bilim ve Sanat Merkezi is a special foundation by the Turkey government which focuses on education of gifted children. In early years students start with general concepts like scientific learning then they assigned to more specialized areas like physics or math.

\end{rSection}

%----------------------------------------------------------------------------------------
%	WORK EXPERIENCE SECTION
%----------------------------------------------------------------------------------------

\begin{rSection}{Experience}

\begin{rSubsection}{Carbon Health}{February 2021 - Present}{Full Stack Developer}{Remote}
\item Mostly backend engineer using Scala \& Kotlin
\item Writing/Maintaining new microservices
\begin{list}{$\cdot$}{\leftmargin=2em}
\itemsep -0.5em \vspace{-0.5em} % Compress items in list together for aesthetics
    \item Using technologies including PostgreSQL, gRPC, SQS/SNS, Elastic Search and many others.
    \item Optimized a long running database read job from 25 hours to ~40 minutes with better use of database indexes and optimizing queries for performance.
    \item Designed and integrated an async workflow into the current system. Instrumented logging and observability tools to monitor problems easily.
    \item Designed and implemented a microservice.    
\end{list}
\item Also working on React Native projects when needed.
\end{rSubsection}

%------------------------------------------------

\begin{rSubsection}{Research and Internship at iVis}{June 2020 - December 2020}{}{Ankara, Turkey}
\item Incremental Packing Research (In progress)
\begin{list}{$\cdot$}{\leftmargin=2em}
\itemsep -0.5em \vspace{-0.5em} % Compress items in list together for aesthetics
    \item Packing of disconnected graphs is an important topic in various areas. There are very efficient algorithms in terms of fullness but these algorithms ignore the initial layout of the graphs. We are trying to come up with a feasible packing algorithm that works incrementally so that user preserves their mental map
\end{list}
\item \href{https://github.com/iVis-at-Bilkent/cytoscape.js-layout-utilities}{Cytoscape Layout Utilities}
\begin{list}{$\cdot$}{\leftmargin=2em}
\itemsep -0.5em \vspace{-0.5em} % Compress items in list together for aesthetics
    \item Formerly, the packing algorithm of this extension was centering to the a fixed point
    \item Added the ability to preserve to the current center of components
    \item Fixed some bugs that would lead the algorithm to crash
    \item For my contributions you can check PR \href{https://github.com/iVis-at-Bilkent/cytoscape.js-layout-utilities/pull/19}{\#19}
\end{list}
\item \href{https://github.com/onsah/cytoscape.js-context-menus}{Cytoscape Context Menus}
\begin{list}{$\cdot$}{\leftmargin=2em}
    \itemsep -0.5em \vspace{-0.5em} % Compress items in list together for aesthetics
    \item Removed jQuery dependency by replacing functionality of jQuery with DOM api or implementing the similar functions
    \item Added recursive submenu feature. Formerly the extension only had top level menu and no submenu. I added nested submenus so that any menu item can have submenus
    \item Configured project with webpack and babel to develop using modern js but deploy in more compatible js version for browsers. Learned about modern js tools such as bundlers and transpilers
\end{list}
\end{rSubsection}

\end{rSection}

%----------------------------------------------------------------------------------------
%	TECHNICAL STRENGTHS SECTION
%----------------------------------------------------------------------------------------

\begin{rSection}{Technical Strengths}

\begin{tabular}{ @{} >{\bfseries}l @{\hspace{6ex}} l }
Computer Languages & Prolog, Haskell, AWK, Erlang, Scheme, ML \\
Protocols \& APIs & XML, JSON, SOAP, REST \\
Databases & MySQL, PostgreSQL, Microsoft SQL \\
Tools & SVN, Vim, Emacs
\end{tabular}

\end{rSection}

%----------------------------------------------------------------------------------------
%	EXAMPLE SECTION
%----------------------------------------------------------------------------------------

%\begin{rSection}{Section Name}

%Section content\ldots

%\end{rSection}

%----------------------------------------------------------------------------------------

\end{document}
