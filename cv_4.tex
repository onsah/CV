%%%%%%%%%%%%%%%%%%%%%%%%%%%%%%%%%%%%%%%%%
% Medium Length Professional CV
% LaTeX Template
% Version 2.0 (8/5/13)
%
% This template has been downloaded from:
% http://www.LaTeXTemplates.com
%
% Original author:
% Trey Hunner (http://www.treyhunner.com/)
%
% Important note:
% This template requires the resume.cls file to be in the same directory as the
% .tex file. The resume.cls file provides the resume style used for structuring the
% document.
%
%%%%%%%%%%%%%%%%%%%%%%%%%%%%%%%%%%%%%%%%%

%----------------------------------------------------------------------------------------
%	PACKAGES AND OTHER DOCUMENT CONFIGURATIONS
%----------------------------------------------------------------------------------------

\documentclass{resume} % Use the custom resume.cls style

\usepackage[left=0.75in,top=0.6in,right=0.75in,bottom=0.6in]{geometry} % Document margins
\usepackage{hyperref}

\name{Onur Şahin} % Your name
\address{Software Engineer} % Your address
\address{sahinonur2000@hotmail.com \\ https://github.com/onsah} % Your phone number and email

\begin{document}

%----------------------------------------------------------------------------------------
%	EDUCATION SECTION
%----------------------------------------------------------------------------------------

\begin{rSection}{Education}

{\bf Bilkent University, Turkey} \hfill {\em June 2021} \\ 
B.S. in Computer Science \\
Full Scholarship \\
Overall GPA: 3.49/4.00

{\bf Bilim ve Sanat Merkez, Turkey} \hfill {\em 2006-2016} \\ 
Bilim ve Sanat Merkezi is a special foundation by the Turkey government which focuses on education of gifted children. In early years students start with general concepts like scientific learning then they assigned to more specialized areas like physics or math.

\end{rSection}

%----------------------------------------------------------------------------------------
%	WORK EXPERIENCE SECTION
%----------------------------------------------------------------------------------------

\begin{rSection}{Experience}

\begin{rSubsection}{Carbon Health}{February 2021 - Present}{Full Stack Developer}{Remote}
\item Mostly backend using Scala \& Kotlin
\item Writing/Maintaining new microservices
\begin{list}{$\cdot$}{\leftmargin=2em}
\itemsep -0.5em \vspace{-0.5em} % Compress items in list together for aesthetics
    \item Using technologies including PostgreSQL, gRPC, SQS/SNS, Elastic Search and many others.
    \item Optimized a long running database read job from 25 hours to 40 minutes with better use of database indexes and optimizing queries to reduce time complexity.
    \item Designed and integrated an async workflow into the current system. Instrumented logging and observability tools to monitor problems easily.
    \item Designed and implemented a microservice.    
\end{list}
\item Also working on React Native projects when needed.
\end{rSubsection}

%------------------------------------------------

\begin{rSubsection}{Research and Internship at iVis}{June 2020 - December 2020}{}{Ankara, Turkey}
\item Incremental Packing Research
\begin{list}{$\cdot$}{\leftmargin=2em}
\itemsep -0.5em \vspace{-0.5em} % Compress items in list together for aesthetics
    \item Packing of disconnected graphs is an important topic in various areas. There are very efficient algorithms in terms of fullness but these algorithms ignore the initial layout of the graphs. We are trying to come up with a feasible packing algorithm that works incrementally so that user preserves their mental map
\end{list}
\item \href{https://github.com/iVis-at-Bilkent/cytoscape.js-layout-utilities}{Cytoscape Layout Utilities}
\begin{list}{$\cdot$}{\leftmargin=2em}
\itemsep -0.5em \vspace{-0.5em} % Compress items in list together for aesthetics
    \item Formerly, the packing algorithm of this extension was centering to the a fixed point
    \item Added the ability to preserve to the current center of components
    \item Fixed some bugs that would lead the algorithm to crash
    \item For my contributions you can check PR \href{https://github.com/iVis-at-Bilkent/cytoscape.js-layout-utilities/pull/19}{\#19}
\end{list}
\end{rSubsection}

\end{rSection}

\begin{rSection}{Projects}

\begin{rSubsection}{(Conference Paper) Augmenting Code Review Experience Through Visualization}{September 2020-July 2021}{}{}
    \item \url{https://ieeexplore.ieee.org/abstract/document/9604852}
    \item Implemented a prototype tool that visualizes code difference between two versions of a Java project.
    \item Tool is based on OOP concepts like packages, classes and fields and extensible to other classes. And extensible to other OOP languages.
    \item Paper is presented on \href{https://vissoft.info/2021/}{VISSOFT 2021} conference.
\end{rSubsection}

\begin{rSubsection}{Contribution to the Jakt programming language}{}{}{}
    \item Contribued to the typechecker and parser of the language.
    \item \href{https://github.com/SerenityOS/jakt/pulls?q=is%3Apr+author%3Aonsah}{List of PRs that I opened.}
\end{rSubsection}

\begin{rSubsection}{\href{https://github.com/onsah/Flux_rs/}{Flux programming language} (Personal Project)}{}{}{}
    \item Fully featured scripting language implementation.
    \item Bytecode interpreted.
    \item Completely written from scratch without using external libraries for parsing, compiling, etc.
    \item Implemented using Rust.
\end{rSubsection}
    
\end{rSection}

%----------------------------------------------------------------------------------------
%	TECHNICAL STRENGTHS SECTION
%----------------------------------------------------------------------------------------

\begin{rSection}{Technical Strengths}

\begin{tabular}{ @{} >{\bfseries}l @{\hspace{6ex}} l }
Areas & Parsers, Lexers, CRUD applications, Event driven programming \\
Computer Languages & Scala, Rust, Java, Kotlin, OCaml \\
Protocols \& APIs & REST, gRPC \\
Databases & PostgreSQL \\
\end{tabular}

\end{rSection}

\begin{rSection}{Other}
    \item Placed 156th among 1.5 million student in Turkey University Entrance Exam (Ygs).
    \item The textbooks that I read:
    \begin{list}{$\cdot$}{\leftmargin=2em}
    \itemsep -0.5em \vspace{-0.5em} % Compress items in list together for aesthetics
    \item Modern Compiler Construction in ML (Currently Reading)
    \item Programming – Principles and Practice Using C++ - Bjarne Strousburp
    \item The Rust Programming Language - Steve Klabnik and Carol Nichols
    \end{list}
    \item I play electro guitar for two years now. I like to cover my favorite songs and write my own instrumental pieces.
\end{rSection}

%----------------------------------------------------------------------------------------
%	EXAMPLE SECTION
%----------------------------------------------------------------------------------------

%\begin{rSection}{Section Name}

%Section content\ldots

%\end{rSection}

%----------------------------------------------------------------------------------------

\end{document}
